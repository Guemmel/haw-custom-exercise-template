
%%%%%%%%%%%%%%%%%%%%%%%%%%%%%%%%%%%%%%%%%%%%%%%%%%%%%%%%%%%%%%%%%%%%%
% Packages
%%%%%%%%%%%%%%%%%%%%%%%%%%%%%%%%%%%%%%%%%%%%%%%%%%%%%%%%%%%%%%%%%%%%%

\usepackage{fancyhdr} % Header and Foote
\usepackage{lastpage}
\usepackage[utf8]{inputenc}
\usepackage[ngerman]{babel}
\usepackage{microtype}
\usepackage{multicol}
\usepackage{ulem}

\usepackage[
    bookmarks,
    bookmarksnumbered,
    bookmarksopen=true,
    bookmarksopenlevel=1,
    colorlinks=true,
    % Links are colored according to print mode (LinkColor from Format.tex)
    linkcolor=LinkColor,
    anchorcolor=LinkColor,
    citecolor=LinkColor,
    filecolor=LinkColor,
    menucolor=LinkColor,
    urlcolor=LinkColor,
    backref, % References link back to citations
    linktoc=all,
    pdfpagelabels=true
]{hyperref}

%%%%%%%%%%%%%%%%%%%%%%%%%%%%%%%%%%%%%%%%%%%%%%%%%%%%%%%%%%%%%%%%%%%%%
% Colors
%%%%%%%%%%%%%%%%%%%%%%%%%%%%%%%%%%%%%%%%%%%%%%%%%%%%%%%%%%%%%%%%%%%%%

\usepackage{xcolor}
\definecolor{haw}{HTML}{003CA0}
\definecolor{LinkColor}{HTML}{003CA0} % Color for links

\usepackage{sectsty}
\sectionfont{\color{haw}} % Set section color
\subsectionfont{\color{haw}} % Set subsection color
\subsubsectionfont{\color{haw}} % Set subsubsection color

%%%%%%%%%%%%%%%%%%%%%%%%%%%%%%%%%%%%%%%%%%%%%%%%%%%%%%%%%%%%%%%%%%%%%
% Page Layout
%%%%%%%%%%%%%%%%%%%%%%%%%%%%%%%%%%%%%%%%%%%%%%%%%%%%%%%%%%%%%%%%%%%%%

\usepackage{geometry}
\geometry{a4paper,left=25mm,right=25mm,top=25mm,bottom=35mm}

\usepackage{helvet}
\renewcommand{\familydefault}{\sfdefault}
\renewcommand{\arraystretch}{1.3} % increases row height (default is 1.0)

\usepackage{setspace}
\onehalfspacing
\setlength{\parindent}{0pt} % Disable paragraph indentation globally


%%%%%%%%%%%%%%%%%%%%%%%%%%%%%%%%%%%%%%%%%%%%%%%%%%%%%%%%%%%%%%%%%%%%%
% Theorem
%%%%%%%%%%%%%%%%%%%%%%%%%%%%%%%%%%%%%%%%%%%%%%%%%%%%%%%%%%%%%%%%%%%%%

\usepackage[framemethod=TikZ]{mdframed}
\usepackage{amsthm}
\usepackage{thmtools}
% \usepackage[framemethod=TikZ]{mdframed} % Already loaded in Document.tex
\theoremstyle{definition}

\declaretheoremstyle[
    headfont=\bfseries\sffamily\color{haw!70!black},
    bodyfont=\normalfont,
    mdframed={
        linewidth=2pt,
        leftline=true, rightline=false, topline=false, bottomline=false,
        linecolor=haw, backgroundcolor=haw!5,
        skipabove=0.5cm % Add this for space before the mdframed box
    },
    headformat={\NAME\ \NUMBER\ \textmd{--}\nobreakspace},
    headpunct={}
]{hawbox}

\declaretheorem[style=hawbox, name=Aufgabe]{Aufgabe}

%%%%%%%%%%%%%%%%%%%%%%%%%%%%%%%%%%%%%%%%%%%%%%%%%%%%%%%%%%%%%%%%%%%%%
% TABLE STUFF
%%%%%%%%%%%%%%%%%%%%%%%%%%%%%%%%%%%%%%%%%%%%%%%%%%%%%%%%%%%%%%%%%%%%%
\usepackage{csvsimple}
\usepackage{float}
\usepackage{etoolbox} % Added for advanced command logic
\restylefloat{table}


% 1. Caption
% 2. Label
% 3. Table file
\newcommand{\mtable}[3]{
    \begin{table}[H]
        \centering
        \singlespacing
        \input{#3}
        % \caption{#1}
        \label{tab:#2}
    \end{table}
}

%%%%%%%%%%%%%%%%%%%%%%%%%%%%%%%%%%%%%%%%%%%%%%%%%%%%%%%%%%%%%%%%%%%%%
% Graphs
%%%%%%%%%%%%%%%%%%%%%%%%%%%%%%%%%%%%%%%%%%%%%%%%%%%%%%%%%%%%%%%%%%%%%
\usepackage{pgfplots}

% Insert graph: caption, label, file
\newcommand{\graph}[3]{%
        \begin{figure}[H]
            \centering
            \input{#3}
            % \caption{#1}
            \label{graph:#2}
        \end{figure}
}